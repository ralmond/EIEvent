\documentclass{article}
\usepackage{amsmath,graphicx}
%\usepackage{rgadefs}
%\usepackage{rgafigs}
\usepackage{indentfirst}
\usepackage{url}
\usepackage{apacite}
\usepackage{alltt}
\usepackage{algorithmicx}
\usepackage{algorithm}
\floatname{algorithm}{Example}

\def\logit{\mathop{\rm logit}\nolimits}
\def\mean{\mathop{\rm mean}\nolimits}
\def\Var{\mathop{\rm Var}\nolimits}
%
\def\cat{\mathop{\rm cat}\nolimits}
\def\Dirichlet{\mathop{\rm Dirichlet}\nolimits}
\def\lognormal{\mathop{\rm lognormal}\nolimits}
%% BF greek
\def\bfalpha{\boldsymbol\alpha}
\def\bfbeta{\boldsymbol\beta}
\def\bfgamma{\boldsymbol\gamma}
\def\bfdelta{\boldsymbol\delta}
\def\bfepsilon{\boldsymbol\epsilon}
\def\bfzeta{\boldsymbol\zeta}
\def\bfeta{\boldsymbol\eta}
\def\bftheta{\boldsymbol\theta}
\def\bfiota{\boldsymbol\iota}
\def\bfkappa{\boldsymbol\kappa}
\def\bflambda{\boldsymbol\lambda}
\def\bfmu{\boldsymbol\mu}
\def\bfnu{\boldsymbol\nu}
\def\bfxi{\boldsymbol\xi}
\def\bfpi{\boldsymbol\pi}
\def\bfrho{\boldsymbol\rho}
\def\bfsigma{\boldsymbol\sigma}
\def\bftau{\boldsymbol\tau}
\def\bfupsilon{\boldsymbol\upsilon}
\def\bfphi{\boldsymbol\phi}
\def\bfchi{\boldsymbol\chi}
\def\bfpsi{\boldsymbol\psi}
\def\bfomega{\boldsymbol\omega}
\def\bfOmega{\boldsymbol\Omega}

%--------------------------------------------
%JSON listing mode, from
% https://tex.stackexchange.com/questions/83085/how-to-improve-listings-display-of-json-files#83100

\usepackage{listings}
\usepackage{xcolor}

\colorlet{punct}{red!60!black}
\definecolor{background}{HTML}{EEEEEE}
\definecolor{delim}{RGB}{20,105,176}
\colorlet{numb}{magenta!60!black}

\lstdefinelanguage{json}{
    basicstyle=\normalfont\ttfamily,
    numbers=left,
    numberstyle=\scriptsize,
    stepnumber=1,
    numbersep=8pt,
    showstringspaces=false,
    breaklines=true,
    frame=lines,
    backgroundcolor=\color{background},
    literate=
     *{0}{{{\color{numb}0}}}{1}
      {1}{{{\color{numb}1}}}{1}
      {2}{{{\color{numb}2}}}{1}
      {3}{{{\color{numb}3}}}{1}
      {4}{{{\color{numb}4}}}{1}
      {5}{{{\color{numb}5}}}{1}
      {6}{{{\color{numb}6}}}{1}
      {7}{{{\color{numb}7}}}{1}
      {8}{{{\color{numb}8}}}{1}
      {9}{{{\color{numb}9}}}{1}
      {:}{{{\color{punct}{:}}}}{1}
      {,}{{{\color{punct}{,}}}}{1}
      {\{}{{{\color{delim}{\{}}}}{1}
      {\}}{{{\color{delim}{\}}}}}{1}
      {[}{{{\color{delim}{[}}}}{1}
      {]}{{{\color{delim}{]}}}}{1},
}


%************************* Title & Authors ******************************
\title{\large{\bf Rule of Evidence for Parsing Event Logs}}
\author{Russell G. Almond\\ Florida State University\\ }
%******************************* Abstract *****************************

\begin{document}

  \maketitle
\begin{abstract}
A common need in game- and simulation-based assessments is to parse a
log made up of many event records either as the assessment is running
or afterwards to score this process.  The end result of this
processing is define a collection of \textit{observable} outcome
variables which are then used for summary scoring, task based feedback
or other analytics.  In evidence-centered assessment design (ECD)
these observable are defined by \textit{rules of evidence}.  This
manual describes the specification for a language, called
\textit{EI-Event} (evidence identification from events) which
specifies rules of evidence in a JavaScript query format, similar to
that used by the Mongo database.
\end{abstract}
\vspace{1cm} \noindent{Key words}: Event Logs, JSON, Evidence
Identification, Test Scoring


  \section{Introduction}

  \citeA{Proc4} defined a generalized model for assessment that
  consists of four processes:
  \begin{description}
    \item[Presentation Process (PP)]{Presents stimulus material to examinee
      and captures work product.}
    \item[Evidence Identification (EIP)]{Extracts evidence in the form of
      observed outcome variables from the raw work product.}
    \item[Evidence Accumulation (EAP)]{Combines observable outcomes from
      several tasks to produce statistics about examinee
      competencies.}
    \item[Activity Selection (ASP)]{Assigns the next task based on current
      competency estimates and previous observations.}
  \end{description}
  This paper focuses on the second of these, the evidence
  identification process.  In many cases, evidence identification is
  almost trivially simple.  For multiple choice tests it simply
  matches the work product (the selected answer) to the key.  Many
  short answer question types also have simple pattern matching keys.
  However, complex work products, particularly event logs from
  simulators, require much more complex EIP.

  \citeA{OnTheStructure} defined an evidence model with two parts:
  the statistical model or \textit{weights of evidence}---which
  correspond to the EAP,---and the \textit{rules of evidence}---which
  corresponds to the EIP.  Although the term \textit{rules of
    evidence} is a pun on the legal term, literal rules of evidence
  were used in HyDRIVE \cite{HyDRIVE}, one of the exemplars used in
  building the original evidence-centered design framework.  HyDRIVE
  was a simulation of maintaining the hydraulics system of the F-15
  aircraft.  The evidence identification system for HyDRIVE was
  written in Prolog, and consisted of ``rules'' that would fire if
  certain conditions were met.

  For game and simulation based assessments, this rule-based approach
  to defining the evidence identification process works fairly well.
  Conceptually a rule looks something like:
  \begin{equation*}
    \texttt{ WHEN \emph{context} IF \emph{condition} THEN
      \emph{observable} = \emph{value}}.
  \end{equation*}
  Here \emph{context} is the context in which the evidence is
  gathered; in simple systems this is the task.  In simulator systems
  the task can be an emergent property of the simulator;
  Section~\ref{sec:context} explores this in more detail. The
  \emph{condition} often involves querying the state of the system,
  this in turn requires that the state of the system be monitored.
  Section~\ref{sec:stateMachine} describes the use of a state machine
  to track the state of the simulator.

  This paper defines the EI-Event protocol for handling evidence
  identification.  It processes a collection of events in a log files
  by running a collection of rules.  In particular, it assumes that
  the game or simulator is logging to a learning record store
  (essentially a database) event descriptions that look something
  like:
  
  \begin{algorithm}
    \caption{Generic Event Record, JSON format.}
    \label{json:event}
    \begin{lstlisting}[language=json]
      {
        app:"ecd://coe.fsu.edu/EPLS/AssessmentName",
        uid:"Student/User ID",
        timestamp:"Time at which event occurred",
        verb:"Action Keyword",
        object:"Object Keyword",
        data:{
          field1:"Value",
          field2:["list","of","values"],
          field3:{part1:"complex",part2:"object"}
        }
      }
  \end{lstlisting}
  \end{algorithm}

  The format of Example~\ref{json:event} is JSON (java script object
  notation).  This is list of key--value pairs with the key and value
  separated by a colon (:).  The values can be numeric values,
  strings, date-time objects, arrays (using the square brackets, []),
  or objects (using curly braces, \{\}).  This event format is a
  simplified version of the xAPI\footnote{The simplification is mainly
    replacing the \emph{verb} and \emph{object} values with strings
    where xAPI uses more complex objects.  In particular, xAPI uses a
    full URL to uniquely define the verb and object, as the same word
    could have different meanings in the context of the application.
    In the simplified version, the vocabulary is defined by the
    application, allowing the message itself to be simpler.} format
  used by Learning Locker \cite{xAPI}.  The first five header fields
  are common to every event while the format of the data field is
  completely open and can contain arbitrarily complex status data.

  The assumption is that the the work product the EIP receives from
  the PP is an ordered sequence of these event records.  (This could
  either be directly sent from the PP to the EIP or the EIP could
  retrieve them on demand from a learning record store.)  The EIP
  processes these one at a time, updating the state of the system.  In
  the end, the EIP must send one (or more) message to the EAP stating
  providing values for the observables seen in a particular context.
  This is what the EAP uses to update the student model.
  
  \section{Tasks and Contexts}
  \label{sec:context}

  \citeA{OnTheStructure} deliberately used the term \textit{task}
  instead of the more familiar \textit{item} to encourage test
  designers to think beyond the common multiple-choice and
  short-answer task types. In 2003, it was obvious that assessments
  using more complex simulation tasks would be an important part of
  the future of assessment.

  \citeA{Proc4} had a more operational definition for \textit{task}:  a
  task was the grain size at which information was passed between the
  four processes.  The PP would present a task to a student and
  send the work product for that task to the EIP.  The EIP would
  process the work product and send the observable for that task to
  the EAP.  The EAP would signal to the ASP that the task was
  complete, and it would select the next task to present and send that
  information to the PP.  Depending on the application, a task could
  be a single item, a group of items with a common stimulus, or a more
  complex constructed response or simulation task.

  \citeA{ECGD} \cite<also>{Mislevy2013} noted that in simulation and
  game-based assessments, 
  tasks do not necessarily come in clean units as in more traditional
  assessments.  Often a task will be a subsequence of events that
  happen in the course of a larger simulation or game.  In this case,
  the term task is no longer quite appropriate because several
  measurement contexts might occur in the course of the player
  completing a single game task.
  
  \subsection{Three Examples}

  To address the more complex situations, in this paper, the unit of
  movement around the four process architecture is termed a
  \textit{context}.  The exact definition of a context depends on the
  assessment application.  This section provides three examples of
  increasing complexity.

  \noindent\textit{Computer Adaptive Test}.  Consider first a computer
  adaptive test similar to ETS's Graduate Record Exam
  (GRE{\textregistered}) as it operated around 2003.  In this
  assessment, the \textit{context} or \textit{task} is equivalent to
  an item.  The ASP selects a single item, which the PP displays, the
  EIP scores and the EAP updates the estimate of student ability.  The
  ASP then tries to select a new item that maximizes information about
  the examinee while balancing content constraints and minimizing item
  exposures.

  Item sets, such as a reading passage followed by several items,
  presented a problem for the item-as-task design.  If one item was
  chosen from a set, then the next couple of items were constrained to
  also be chosen from the same set.  These items could be sub-optimal
  for satisfying information targets or content and exposures control
  constraints.  A better solution would be to make the task for such
  set-based items the item set, so an optimal item set could be
  chosen.

  It is interesting to note that more recent version of the GRE have
  gone from using a single item as the task to using a testlet
  containing 10 items or so as the task.  With this kind of task, it
  is easier to balance information properties, content constraint and
  exposure control as well as to better control context effects of
  items.

  In this example, as in most classical assessment, the context is
  the same as the task---both are engineered by the test designers.

  \noindent\textit{Physics Playground}.  The game \textit{Physics
    Playground} \cite{ShuteVentura2013,KimPhysicsPlayground} is a
  game for teaching knowledge of Newtonian physics with an embedded
  stealth assessment.  The game is level-based: each game level
  is a puzzle where players must maneuver a ball to reach a target
  balloon.  In sketching levels, players draw objects which can be
  used to apply forces to the ball.  In manipulation levels, the
  players manipulate the forces on the ball through the use of
  sliders. As in the computer adaptive assessment, the
  \textit{Physics Playground} task is engineered by the designers.  In
  this case it is the game level.

  Even in a level-based game, the task boundary is not clear.  Is a
  task a single attempt at a game level or all attempts at a game
  level in a playing session?  How are the boundaries of an attempt
  defined?  What if a player restarts the level?  Leaves the level and
  then returns?  The right answer to these questions will depend on
  the purpose of the assessment, but unlike the computer adaptive
  test, there is no submit answer button to define the context
  boundary.

  Note that in \textit{Physics Playground}, there are multiple
  observed outcomes per task.  The game records not only whether the
  level was successfully completed, but also if the player used a
  particular tool to solve a level.  This information must be
  extracted from the game logs by the EIP.

  The \textit{Physics Playground} PP logs events to a Learning Locker
  learning store \cite{LearningLocker}.  These events are recorded as
  JSON objects similar to the one shown in Example~\ref{json:event}.
  Note that this format does not have a field for task or context.  It
  must be inferred from the events.  In particular, certain events in
  the sequence mark the start and the end of the level.  So the EIP
  must infer the context from the information stream and must
  internally keep track of where the context is.

  \noindent\textit{Flight Simulator}.  Consider a person training to
  be a pilot using a flight simulator.  The task in this situation
  might consist of flying the plane from an origin airport to a
  destination airport.  There might be several contexts that emerge
  within this single task.  For example, the takeoff is one task while
  flying the plane after it reaches cruising altitude is another.  If
  a thunderstorm occurs along the route that might change the context
  to handling the weather event, after which the simulator returns to
  the cruising context.  Finally, the approach and the landing might
  also be contexts.

  This example makes clear the need for moving from task to context as
  the unit that goes around the four process loop.  Here many
  variables about the state of the system may be related to the
  definition of the context.

  \subsection{Context as an Emerging Status}

  One way to detect a context change is to keep track of the state of
  the simulator.  When the state changes in such a way as to define a
  new context, then the context changes.  In a complete system, the
  EIP needs two kinds of rules to support this.  First, it needs a
  collection of \textit{context rules} which defines when a context
  changes.  Second, it needs \textit{trigger rules} to indicate when
  it should send the observables for a given context to the EAP (or
  other listeners).  At this point, these are rules in the loose sense
  of \citeA{OnTheStructure}:  requirements for the EIP.
  Section~\ref{sec:Rules} will introduce a notation for operationally
  specifying those rules.

  In addition to taking the role of \textit{task} in the four process
  architecture, contexts play another roll in the EIP.  In particular,
  each context has an associated set of specific evidence rules.  For
  example, evidence rules appropriate for a manipulation level in
  \textit{Physics Playground} are not necessarily appropriate for a
  sketching level.  Tagging rules by the contexts in which they are
  applicable allows the EIP to filter the rule set to just those which
  are interesting and appropriate in the current context.  Note that
  there is also a need here for context sets, as the list of contexts
  for which a particular rule is needed may change as tasks are added
  and removed from the assessment.

  The second and third examples have a certain similarity in the work
  product.  In both cases, the game or simulator log consists of a
  series of events.  (Section~\ref{sub:events} describes a proposed
  structure for those event records.)  To process these events, often
  it is necessary to recreate information about the state of the
  system in the EIP.  Section~\ref{sec:stateMachine} describes a state
  machine architecture.

  \section{State Machine}
  \label{sec:stateMachine}

  The identification of evidence often requires fairly detailed
  information about what is happening in the game.  Often, the best
  evidence is when the player performs an action which changes the
  state of the system.  To assess such evidence, the EIP often needs
  to at least partially recreate the state of the system. The
  architecture for the EI-Event protocol, therefore starts
  with the idea that the EIP is a finite state machine.  Each user
  (player or student) has an associated state object.  These state
  objects have collections of observables, timers and flags which
  change in response to events.  These are described in more detail
  below.

  Note that the state machine approach to EIP is useful even in
  contexts which are not simulations or games.  In particular,
  \citeA{KeystrokeRR} used a finite state machine to analyze keystroke
  timing data.  It was necessary to parse the stream of characters
  recorded by the logger to associate pause events with individual
  words or with between word or sentence spaces.  This is potentially
  a robust approach.

  \subsection{Observables}
  \label{sub:obs}

  Consider the game-based assessment \textit{Physics Playground}.  As
  described above, the context for this application is a game level.
  In each game level the player attempts to move a ball to the target
  (balloon).  The player may draw objects on the screen (sketching
  levels) to reach the target or manipulate physics parameters or the
  strength of blowers producing force (manipulation levels).  For each
  game level, the game engine determines whether or not the game level
  has been passed (solved) and whether a gold or silver coin was
  awarded for the solution.  (Efficient solutions earn gold coins, and
  inefficient solutions earn silver coins.)

  Here are several observables identified in version 2 of
  \textit{Physics Playground}:
  \begin{description}
    \item[Obs1]{What is the maximum value coin (gold, silver or none)
      that the player has earned for this level?}
    \item[Obs2]{Did the player manipulate the gravity slider?}
    \item[Obs3]{How many objects did the player draw?}
    \item[Obs4]{How much time (excluding time spent on learning
      supports) did the player spend on the level?}
    \item[Obs5]{Did the player attempt to draw a springboard?}
  \end{description}

  The first two are fairly simple.  The state machine needs an
  \textit{observable variable} which keeps track of the coin reward
  (Obs1) or the use of the gravity slider (Obs2).  The when an
  appropriate event comes through the log, an evidence rule will
  update the appropriate variable.  Obs3 is similar; in this case, the
  observable is a counter and it is incremented each time an even
  comes through indicating that an object was drawn.

  The timing observable (Obs4) need a bit more work.  The timer must
  be started when the level starts and paused every time the player
  starts using a learning support, resuming when the player returns
  from the learning support.  The next section describes timers in
  more detail.
  
  Detecting springboards is actually quite tricky.  Obs5 requires
  determining if each object that the player has drawn is a
  springboard or not.  However, a springboard has several elements
  (the springboard, the anchor holding one end fixed, and usually a
  weight which gives it elastic potential energy).  Many of these
  might require details of the object stored in the underlying physics
  engine.  In this case, the PP of \textit{Physics Playground}
  contains code to identify agents of motion.  The PP then sends an
  event saying that the agent identification system triggered, which
  the EIP can use to build up observables.

  Although generally, evidence rules are placed in the EIP, in some
  cases it may make sense to place them in other processes.  In the
  example above, it was easier to implement the agent identification
  system in the PP as it had access to the details of the screen
  layout, and object placement and movement.  Similarly in the case
  of a multiple choice test, it is is probably best to have the PP
  send a message containing which option was selected as the event,
  rather than the lower level event of where the student clicked on
  the screen.  The latter requires details of how the item is laid
  out in the screen that are needed in the PP but not the EIP.

  There may also be cases where the EAP can handle the evidence rule.
  For example, consider Obs3 (which is a count) and Obs4 (which is a
  time).  If these are to be used in a model with discrete observables
  (e.g., a Bayesian network) then the continuous or count variable
  might need to be cut into categories (e.g., low, medium, high).
  This might be easier to do in the EAP, especially if the Bayes net
  software can associate ranges of numeric values with states.
  
  \subsection{Flags and Timers}
  \label{sub:flags}

  Not all potential observables need to be reported out of the EIP.
  In general, an observable is reported for one of three reasons:
  \begin{enumerate}
  \item It will be used by the EAP to accumulate a score for the
    student.
  \item It will be used to customize feedback presented to the
    students.
  \item It will be logged to a data base for future analysis
      (for scientific research or to improve the performance
    of the assessment). 
  \end{enumerate}

  Observables taking on one or more of these roles are \textit{final
    observables}, as opposed to \textit{intermediate observables}
  whose role is to aid in the computation of final observable values.
  ETS's e-rater{\textregistered} system \cite{eRater} is a good example.
  The system has low level code that identifies likely grammar, usage,
  mechanics or style errors.  (These low level observables are
  sometimes used to give students feedback on their writing.)  These
  low-level errors are accumulated into count variables providing the
  number of grammar, usage, mechanics and style errors the writer
  made.  These count variables (after suitable normalization), along
  with other variables related to vocabulary and discourse, are fed
  into a regression model to produce the final score.  Note that there
  are observables a number of different levels in this system:  any of
  which could be intermediate or final depending on the requirements
  of the assessment.
  
  Because EIP designers may want to separate out the intermediate and
  final observables, the model design has two collections of
  observables:  the observables collection for storing final
  observables and the flags collection for storing intermediate
  observables.  However, this distinction is not strictly enforced:
  variables in both the flags and observable collections may play the
  role of intermediate observables.

  Timing variable are special.  Looking at the definition of Obs4, it
  is clear that the timer must be started at the beginning of the
  level, paused when the player enters the learning support and resumed
  when the player leaves the learning support.  In order to do this,
  the timer object must support the following operations:
  \begin{description}
  \item[resume]{Continue accumulating time.}
  \item[pause]{Stop accumulating time.}
  \item[reset]{Set the accumulated time to zero.}
  \item[restart]{Combination of reset and resume.}
  \item[value]{Return the time accumulated so far.}
  \end{description}

  These operations can be triggered by rules.  Also, the current value
  of a timer can be copied into an observable or flag variable.
  Finally, the current value of a timer can be queried in the
  condition part of the rule.

  Thus, a state object contains:
  \begin{description}
  \item[uid]{An identifier for the user (student, player)}
  \item[context]{An identifier for the current context.}
  \item[observables]{A collection of observable variables.}
  \item[timers]{A collection of timers.}
  \item[flags]{A collection of other (intermediate observable)
    variables.}
  \end{description}
  

  \subsection{Events}
  \label{sub:events}

  Look once more at the format of the event record in
  Example~\ref{json:event} (reproduced in Example~\ref{json:event1}).
  There are five fields in the header---\texttt{app}, \texttt{uid},
  \texttt{timestamp}, \texttt{verb}, and \texttt{object}---and a
  \texttt{data} container that can contain an arbitrary number of
  fields.  The \texttt{data} field allows the PP to pass arbitrary
  information to the EIP.

  \begin{algorithm}
    \caption{Generic Event Record, JSON format.}
    \label{json:event1}
    \begin{lstlisting}[language=json]
      {
        app:"ecd://coe.fsu.edu/EPLS/AssessmentName",
        uid:"Student/User ID",
        timestamp:"Time at which event occurred",
        verb:"Action Keyword",
        object:"Object Keyword",
        data:{
          field1:"Value",
          field2:["list","of","values"],
          field3:{part1:"complex",part2:"object"}
        }
      }
  \end{lstlisting}
  \end{algorithm}

  The \texttt{verb} and \texttt{object} fields should be keywords
  appropriate to the application.  These fields are taken from the
  xAPI \cite{xAPI} data structure.  The verb and object together
  should define what is happening.  Some example pairs from
  \textit{Physics Playground} include $\{ (\texttt{"snapshot"},
  \texttt{"ball"})$, $(\texttt{"exited"}, \texttt{"level"})$, 
  $(\texttt{"exited"}, \texttt{"learning support"})$, $(\texttt{"identified"}$,
  $\texttt{"game object"}) \}$.  Note that in some cases, the object is
  necessary to distinguish what the verb is operating on (``exited
  learning support'' versus ``exited level'').  In all cases, the
  extra data provides additional information (the position of the ball
  for a snapshot event, the learning support or level exited, what the
  object was identified as).

  The PP and the EIP need to agree on what are the valid values for
  the verb and object fields.  This is the role of the \texttt{app}
  (short for application) field.  Note that this is a long URL-like
  structure giving the name of both the specific vocabulary set and
  the organization that defined the vocabulary.  (This is a
  simplification from the xAPI format, where both the \texttt{verb}
  and \texttt{object} values were more complex objects which provide
  information about the vocabulary used.  In most cases, the 
  application defines the verbs, objects and the expected data
  components for each verb object pair, so longer identifiers for
  verbs and objects are not needed.)

  The remaining two header fields are straightforward.  The
  \texttt{uid} field is matched with the state object, so that the
  system can track many users at the same time.  The
  \texttt{timestamp} is just the time at which the event occurred.

  \subsection{Dot notation}
  \label{sub:dot}

  In general, rules need to be able to reference both the state object
  and the event object.  In particular, they need to reference
  specific fields in those objects.  The JavaScript dot notation
  provides a simple mechanism for doing this.  The dots define
  subcomponents of objects.  Starting with the \texttt{state} and
  \texttt{event} objects, the legal fields are:
  \begin{description}
    \item[\texttt{state.context}]{The current context that the state
      object is in.}
    \item[\texttt{state.oldContext}]{The the context of the state at
      the end of the previous event.  In particular, this can be
      compared to the context to check if the context has changed as a
      result of the event.} 
    \item[\texttt{state.observables.\textit{name}}]{The value of the
      observable named \textit{name}.}
    \item[\texttt{state.timers.\textit{name}}]{The current elapsed
      time of the timer named \textit{name}.}
    \item[\texttt{state.flags.\textit{name}}]{The value of the
      observable named \textit{name}.}
    \item[\texttt{event.verb}]{The verb associated with the current event.}
    \item[\texttt{event.object}]{The object associated with the current event.}
    \item[\texttt{event.timestamp}]{The time at which the event
      occurred.}
    \item[\texttt{event.data.\textit{name}}]{The value of the
      extra data field named \textit{name}.}
  \end{description}

  Note that there potentially can be further dots after the
  \textit{name} of the value.  In particular, if the value of the
  observable, flag or extra data element is itself a compound object,
  the the dot notation can be used to refer to its fields.  Also, in
  the predicate of rules, the dot notation applied to timers allows
  access to the pause, resume and reset operations.

  \section{Rules}
  \label{sec:Rules}

  Conceptually a rule has the following format:  \texttt{WHEN
    \textit{context}, \textit{verb}, \textit{object} IF
    \textit{condition} THEN \textit{predicate}}.  The
  \textit{condition} is logical expression involving the fields of the
  state and the current event.  If this is true, then the
  \textit{predicate} is executed.  The \textit{context}, \textit{verb}
  and \textit{object} are really part of the condition.  Separating
  them into separate components allows the rule set to be indexed by
  those fields, which should improve the speed of execution.
  Example~\ref{json:rule} shows a rule set in JSON format.

  \begin{algorithm}
    \caption{Generic Rule, JSON format.}
    \label{json:Rule}
    \begin{lstlisting}[language=json]
      {
        context:"Context Keyword, Context Group Keyword or ALL",
        verb:"Action Keyword" or ALL,
        object:"Object Keyword or ALL",
        type:"Type Keyword",
        priority:"Numeric Value",
        condition:{...},
        predicate:{...}
      }
  \end{lstlisting}
  \end{algorithm}
  
  The condition (Section~\ref{sub:cond}) and predicate
  (Section~\ref{sub:pred}) are also JSON objects.  They are adapted
  from the query language used by the Mongo database \cite{mongo}.
  The \textit{type} and \textit{priority} fields are used to determine
  the sequence in which rules are run.  Sections~\ref{sub:types}
  and~\ref{sub:ruleSet} describe managing rule sets in more detail.

  \subsection{Types and Timing}
  \label{sub:types}

  For the most part, the predicates of the rules change the state
  object or have other side effects.  This means that the sequence in
  which the rules are executed may have an effect on the final state
  of the system.  When constructing a rule set for a given
  application, the designer needs to be able to provide a partial
  ordering on the rules:  forcing the order when sequence is
  important.

  The EI-Event protocol provides two mechanisms for sequencing rules:
  type and priority.  The type mechanism ensures that rules that
  affect the state of the system are run before rules that report on
  the state.  The priority mechanism controls sequencing within a
  type.

  There are five types of rules, which are run in the following
  sequence:
  \begin{enumerate}
  \item \textit{State Rules}.  These rules should have predicates
    which set flag variables and manipulate timers.  These rules are
    run first.
  \item \textit{Observable Rules}.  These rules should have predicates
    that set observable values, they are run immediately after the
    state rules.
  \item \textit{Context Rules}.  These rules return a new value for
    the \textit{context} field, if this needs to be changed.  These
    are run until either the set of context rules is exhausted or one
    of the rules returns a value other than the current context.  Note
    that the priority is potentially important for determining which
    of several rules govern the new context.
  \item \textit{Trigger Rules}.  These rules have a special predicate
    which sends a message to a process listening to the EIP.  These
    rules are given both the old and new context values as often they
    will trigger when the context changes.
  \item \textit{Reset Rules}.  These rules run only if the context
    changes.  They are used to reset values of various timers and
    flags that should be reset for the new context.
  \end{enumerate}

  As mentioned above, the context, verb, and object for an implicit
  part of the condition of the rule.  These can take either a specific
  value, as defined by the application vocabulary, or the special
  keyword \texttt{All} indicating that this rule should be run no
  matter what the value of the corresponding field in the event or
  state.  The context keyword can also be a context group defining a
  set of contexts to which this rule applies.  Thus, for each event
  the EIP checks five collections of rules, each collection
  corresponding to the current context, verb and object as well as the
  type representing the phase of the event processing.

  There is still a potential for sequence issues within each of these
  collections.\footnote{My recollection from trying to program in
    Prolog is that these sequencing issues were often the hardest part
    of a program to debug.}  This is where the priority mechanism
  comes into play. The priority should be a positive integer.  The
  rules are run in order of priority, lowest numbers going first.
  Ties are broken arbitrarily (and possibly in an implementation
  dependent way).  By adjusting the priorities, the designer should be
  able to avoid potential conflicts and race conditions.

  \subsection{Conditions}
  \label{sub:cond}

  The condition notation is adapted from the query documents used by
  the Mongo database \cite{mongo}.  The basic form is shown in
  Example~\ref{json:cond}.  The field should be
  either a field of the state or the event, using the notation
  described in Section~\ref{sub:dot}.  The simplest form has the value
  as a simple value.  The condition is satisfied if all of the fields
  have the indicated values, and not satisfied if one of them does
  not.  Fields of the event or state which are not mentioned in the
  query document are ignored.

  \begin{algorithm}
    \caption{Basic Condition Query Document}
    \label{json:cond}
    \begin{lstlisting}[language=json]
      condition:{
        <field1>:<value1>,
        <field2>:<value2>,
        ...
      }
    \end{lstlisting}
  \end{algorithm}

  Instead of a simple value, the value part of the query document can
  be a more complex JSON object (enclosed in curly braces).  The
  simplest version has the format:  \texttt{\{ \textit{field1}:\{
    \$\textit{op}:\textit{value1} \}, ... \}}.  The simple operators
  are: \texttt{\$eq} (equality, usually omitted), \texttt{\$ne} (not
  equals), \texttt{\$gt} (greater than), \texttt{\$gte} (greater than
  or equals), \texttt{\$lt} (less than), and \texttt{\$lte}.  Note
  that these can be combined.  For example
  \verb|{ state.flag.objCount: { $gt:5, $lt:10}}| returns true when
  the \texttt{objCount} flag is between 5 and 10.

  A slightly more complex version uses the form \texttt{\{ \textit{field1}:\{
    \$\textit{op}:[\textit{value1a}, \textit{value1b},...] \},
    ... \}}.  Here \textit{op} should be either \texttt{\$in} or
  \texttt{\$nin} (not in).  These conditions are satisfied if the
  value of the field is (not) in the list.  A shortcut is available
  for the \texttt{\$in} operator:  the expression \texttt{\{
    \textit{field1}: [\textit{value1a}, \textit{value1b},...], ... \}}
  is equivalent to more complicated version using \texttt{\$in}.

  Three special operators---\texttt{\$exists}, \texttt{\$isnull}, and
  \texttt{\$isna}---test the state of the field.  The first tests if
  the field exists, the second tests for a null value for the field
  and the third tests for a not-applicable or not-a-number value.  In
  each case, the value should be either \texttt{true} or
  \texttt{false}.  If true, the condition is satisfied if the field
  has the appropriate state, and if false, then if the field does not
  have that state.

  The \texttt{\$not}, \texttt{\$and}, and \texttt{\$or} operators can
  be used to build more complex queries.  The value for \texttt{\$not}
  should be another query document; the not query is satisfied if the
  inner query is not satisfied.  The \texttt{\$and} and \texttt{\$or}
  operators take an array (enclosed in \texttt{[]}) of query
  documents.  These are satisfied if all (or any) of the query
  documents in the array are satisfied.

  The \texttt{\$any} and \texttt{\$all} operators are used when the
  value being tested is an array.  The \texttt{\$any} test is
  satisfied if any of the elements of the array satisfy the test
  and the \texttt{\$all} test is satisfied only if all of the
  elements of the array satisfy the test.
  
  The last two operators produce implementation dependent extensions.
  The first is \texttt{\$regex}.  For this one the value should be a
  regular expression and the field should normally contain a string
  value.  This is implementation dependent because different
  implementations will likely use the host regular expression engine,
  many of which have slight variations.  (Simple regular expressions
  will likely work well in most implementations, but Perl or other
  extensions may or may not be supported.) 

  The final operator is \texttt{\$where}, which actually appears as a
  field and not a value in the query document.  The value should be a
  string giving the name of a function (with arguments \texttt{state}
  and \texttt{event}) which evaluates the state and event and returns
  true or false.  This allows the handling of cases which cannot be
  easily handled by the query language (at the cost of being
  implementation dependent).  It is strongly recommended that these
  query functions have not side effects (except for logging used when
  debugging).  

  \subsection{Predicates}
  \label{sub:pred}

  The predicate takes the generic form shown in
  Example~\ref{json:pred}. The big difference is update operators,
  which control how the fields are modified.  Fields should reference
  fields of the state object (dot notation beginning with
  \texttt{state}), although values can reference fields of the
  events. 

  \begin{algorithm}
    \caption{Basic Predicate Update Document}
    \label{json:pred}
    \begin{lstlisting}[language=json]
      predicate:{
        <update operator1>:{ <field1>:<value1>, ... },
        <update operator2>:{ <field2>:<value2>, ... },
        ...
      }
    \end{lstlisting}
  \end{algorithm}

  The simplest update operator is the \texttt{\$set} operator.  In
  this case the field is set to the value (or if the value references
  an event or flag variable, to the value of that field).  If the
  state object does not have a flag or observable of that name, it
  will be created. 

  The \texttt{\$unset} operator is an inverse.  For this the
  \textit{value} should be one of \texttt{"NULL"}, \texttt{"NA"}, or
  \texttt{"Delete"}.  The first two set the values to NULL and NA (not
  applicable) respectively.  The rules for these values are
  implementation dependent.  The third removes the flag or observable
  from the state object.

  The \texttt{\$incr}, \texttt{\$decr}, \texttt{\$mult},
  \texttt{\$div}, \texttt{\$min}, and \texttt{\$max} operators modify
  the existing field (which should be numeric) based on the operator
  and the argument value.  The first four add, subtract, multiply or
  divide the value of the field by the argument value and set the
  field to the result.   The \texttt{\$min} and \texttt{\$max}
  operators set the field to the largest or smallest of the current
  value and the argument argument value.

  The \texttt{\$addToSet} and \texttt{\$pullFromSet} operate on fields
  which are arrays.  In the first case, the argument value is added to
  the set (if it is not already present), and in the second case it is
  removed from the set.

  The \texttt{\$push} and \texttt{\$pop} operators also operate on
  array fields; however, they treat the arrays like stacks and not
  sets.  The \texttt{\$push} operator adds the argument value to the
  front of the list, and the \texttt{\$pop} operator removes the first
  element, or if the value is a positive integer, it removes the that
  many elements from the front of the stack.

  Timers are treated specially by the \texttt{\$set} operator.  In
  particular, there are two fields of the timer object which can be
  set, \texttt{state.timers.\textit{name}.value} and
  \texttt{state.timers.\textit{name}.running}.  The latter value can
  be set to \texttt{true} or \texttt{false} which causes it to resume
  or pause respectively.  The \texttt{.value} field of the timer sets
  the current time; this is assumed if the field is omitted.  This can
  be treated like a numeric value with the time in seconds, or it can
  be set as an object of the form \texttt{\{time:\textit{number},
    units:\textit{unit}\}}, where \textit{unit} is one of
  \texttt{"sec"}, \texttt{"min"}, \texttt{"hr"}, \texttt{"day"} and so
  forth.  Example~\ref{json:time} gives a few examples:
  
  \begin{algorithm}
    \caption{Setting Timers}
    \label{json:time}
    \begin{lstlisting}[language=json]
      predicate:{
        //Set pause learning support timer
        $set:{ state.timers.learnSup.running: false},
        //Resume level timer
        $set:{ state.timers.level.running:true},
        //Restart last play timer
        $set:{ state.timers.lastPlay.value: 0,
               state.timers.lastPlay.running: true},
        //Add 1 minute to the penalized time timer.
        $incr:{ state.timers.penTime : {time:1,
                                        units:"min"}},
        ...
      }
    \end{lstlisting}
  \end{algorithm}
  
  Finally, the \texttt{\$setCall} operator provides an implementation
  dependent extension mechanism.  The \textit{value} should be a
  string giving the name of a function with three arguments:  the name
  of the field being set, the state object and the event object.  The
  named field will be set to the value returned by this function.

  \subsection{Special Predicates for Context and Trigger Rules}
  \label{sub:specRule}

  The context and trigger rules need special treatment.  Context
  rules, if they fire return a string value naming the context.
  Consequently, the \texttt{predicate} component of
  Example~\ref{json:pred} is replaced by a new \texttt{context}
  component.  This should have a single value.  It can be one of three
  things:  (1)~a string value for the new context, (2)~the name of a
  field which provides the value for the new context, or (3)~an
  expression of the form \texttt{\{\$call:\textit{funname}\}} which
  calls a program of the given name in the implementation language.
  This function should have two arguments, the state and the event,
  and return a string providing the name of the new context.
  
  The role of the trigger rule is to send a message.  The basic output
  message format looks like Example~\ref{json:message}.  These are
  called \texttt{Proc4 messages} because the basic format came from an
  early implementation of the four process architecture \cite{Proc4}.
  When the message is sent, the \texttt{app} and \texttt{uid} values
  are set from the \texttt{state} object.  The \texttt{timestamp}
  field is also set from the current time.  The \texttt{context} field
  is often the value of \texttt{state.oldContext}, although that might
  be overridden.

  \begin{algorithm}
    \caption{Proc4 Message Format}
    \label{json:send}
    \begin{lstlisting}[language=json]
      {
        app:<appid>,
        uid:<uid>,
        context:<contextVariable>,
        message:<message>,
        timestamp:<date-time>,
        data:{<field1>:<value1>, <field2>:<value2>, ...}
      }
    \end{lstlisting}
  \end{algorithm}


  Consequently, the \texttt{predicate} component of a trigger rule is
  replaced with the \texttt{send} component, which specifies parts of
  the message.  Example~\ref{json:send} provides the basic layout.
  The \texttt{message} field is required and should be a string.  If
  the \texttt{context} field is omitted, it will default to the
  current value of \texttt{state.oldContext}.  If the \texttt{data}
  field is omitted, it will default to all of the observables.

  \begin{algorithm}
    \caption{Special Send Predicate for Trigger Rules}
    \label{json:send}
    \begin{lstlisting}[language=json]
      send:{
        message:<message>,
        listeners:<nameOrList>,
        context:<contextVariable>,
        data:{<field1>:<value1>, <field2>:<value2>, ...}
      }
    \end{lstlisting}
  \end{algorithm}

  The \texttt{listeners} field requires more explanation.  A number of
  other processes can register as listeners to messages send from the
  EIP.  Normally, the EAP is a registered listener, but there might be
  other processes involved in logging, feedback generation and
  reporting, or event the ASP which want to listen to the messages.
  If this field is omitted, then the message is sent to all
  listeners.  If a name of a specific listener or an array of listener
  names is supplied, then the message will be sent only two those
  listeners.  This might be useful for error or debugging messages.

  \subsection{Rule Set Maintenance}
  \label{sub:ruleSet}

  Queries, adding rules to set.  Removing rules from sets.

  Defining context groups.

  \section{Examples}
  \label{sec:examples}

  Please help me make some examples.
  
  \section{Software}
  \label{sec:software}

  The SVN repository for the current (very incomplete) implementation
  in R is available at
  \url{https://pluto.coe.fsu.edu/svn/common/EIEvent/}.

  \bibliographystyle{apacite}
  \bibliography{EIRefs}
  
\end{document}
