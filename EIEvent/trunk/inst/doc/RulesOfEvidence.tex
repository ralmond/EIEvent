\documentclass{article}
\usepackage{amsmath,graphicx}
%\usepackage{rgadefs}
%\usepackage{rgafigs}
\usepackage{indentfirst}
\usepackage{url}
\usepackage{apacite}
\usepackage{alltt}
\usepackage{algorithmicx}

\def\logit{\mathop{\rm logit}\nolimits}
\def\mean{\mathop{\rm mean}\nolimits}
\def\Var{\mathop{\rm Var}\nolimits}
%
\def\cat{\mathop{\rm cat}\nolimits}
\def\Dirichlet{\mathop{\rm Dirichlet}\nolimits}
\def\lognormal{\mathop{\rm lognormal}\nolimits}
%% BF greek
\def\bfalpha{\boldsymbol\alpha}
\def\bfbeta{\boldsymbol\beta}
\def\bfgamma{\boldsymbol\gamma}
\def\bfdelta{\boldsymbol\delta}
\def\bfepsilon{\boldsymbol\epsilon}
\def\bfzeta{\boldsymbol\zeta}
\def\bfeta{\boldsymbol\eta}
\def\bftheta{\boldsymbol\theta}
\def\bfiota{\boldsymbol\iota}
\def\bfkappa{\boldsymbol\kappa}
\def\bflambda{\boldsymbol\lambda}
\def\bfmu{\boldsymbol\mu}
\def\bfnu{\boldsymbol\nu}
\def\bfxi{\boldsymbol\xi}
\def\bfpi{\boldsymbol\pi}
\def\bfrho{\boldsymbol\rho}
\def\bfsigma{\boldsymbol\sigma}
\def\bftau{\boldsymbol\tau}
\def\bfupsilon{\boldsymbol\upsilon}
\def\bfphi{\boldsymbol\phi}
\def\bfchi{\boldsymbol\chi}
\def\bfpsi{\boldsymbol\psi}
\def\bfomega{\boldsymbol\omega}
\def\bfOmega{\boldsymbol\Omega}

%--------------------------------------------
%JSON listing mode, from
% https://tex.stackexchange.com/questions/83085/how-to-improve-listings-display-of-json-files#83100

\usepackage{listings}
\usepackage{xcolor}

\colorlet{punct}{red!60!black}
\definecolor{background}{HTML}{EEEEEE}
\definecolor{delim}{RGB}{20,105,176}
\colorlet{numb}{magenta!60!black}

\lstdefinelanguage{json}{
    basicstyle=\normalfont\ttfamily,
    numbers=left,
    numberstyle=\scriptsize,
    stepnumber=1,
    numbersep=8pt,
    showstringspaces=false,
    breaklines=true,
    frame=lines,
    backgroundcolor=\color{background},
    literate=
     *{0}{{{\color{numb}0}}}{1}
      {1}{{{\color{numb}1}}}{1}
      {2}{{{\color{numb}2}}}{1}
      {3}{{{\color{numb}3}}}{1}
      {4}{{{\color{numb}4}}}{1}
      {5}{{{\color{numb}5}}}{1}
      {6}{{{\color{numb}6}}}{1}
      {7}{{{\color{numb}7}}}{1}
      {8}{{{\color{numb}8}}}{1}
      {9}{{{\color{numb}9}}}{1}
      {:}{{{\color{punct}{:}}}}{1}
      {,}{{{\color{punct}{,}}}}{1}
      {\{}{{{\color{delim}{\{}}}}{1}
      {\}}{{{\color{delim}{\}}}}}{1}
      {[}{{{\color{delim}{[}}}}{1}
      {]}{{{\color{delim}{]}}}}{1},
}


%************************* Title & Authors ******************************
\title{\large{\bf Rule of Evidence for Parsing Event Logs}}
\author{Russell G. Almond\\ Florida State University\\ }
%******************************* Abstract *****************************

\begin{document}

  \section{Introduction}

  \citeA{Proc4} defined a generalized model for assessment that
  consists of four processes:
  \begin{describe}
    \item[Presentation Process (PP)]{Presents stimulus material to examinee
      and captures work product.}
    \item[Evidence Identification (EIP)]{Extracts evidence in the form of
      observed outcome variables from the raw work product.}
    \item[Evidence Accumulation (EAP)]{Combines observable outcomes from
      several tasks to produce statistics about examinee
      competencies.}
    \item[Activity Selection (AS)]{Assigns the next task based on current
      competency estimates and previous observations.}
  \end{describe}
  This paper focuses on the second of these, the evidence
  identification process.  In many cases, evidence identification is
  almost trivially simple.  For mulitple choice tests it simiply
  matches the work product (the selected answer) to the key.  Many
  short answer question types also have simple pattern matching keys.
  However, complex work products, paraticularly event logs from
  simulators, require much more complex EIP.

  \citeA{OnTheStructure} defined an evidence model with two parts:
  the staticial model or \textit{weights of evidence}---which
  correspond to the EAP,---and the \textit{rules of evidence}---which
  corresponds to the EIP.  Although \textit{rules of evidence} is 
  obviously a pun on the legal term, literal rules of evidence were
  used in HyDRIVE \cite{HyDRIVE}, one of the exemplars used in
  building the original evidence-centered design framework.  HyDRIVE
  was a simulation of maintaining the hydraulics system of the F-15
  aircraft.  The evidence identification system was written in
  prologue, and consisted of ``rules'' that would fire if certain
  conditions were met.

  For game and simulation based assessments, this rule-based approach
  to defining the evidence identification process works fairly well.
  Conceptually a rule looks something like:
  \begin{equation*}
    \begin{alltt} WHEN \emph{context} IF \emph{condition} THEN
      \emph{observable} = \emph{value}.
    \end{alltt}
  \end{equation*}
  Here \emph{context} is the context in which the evidence is
  gathered; in simple systems this is the task.  In simulator systems
  the task can be an emergent property of the simulator;
  Section~\ref{sec:context} explores this in more detail. The
  \emph{condition} often involves querying the state of the system,
  this in turn requires that the state of the system be monitored.
  Section~\ref{sec:stateMachine} describes the use of a state machine
  to track the state of the simulator.

  This paper looks specifically at the problems of create rule sets
  for processing log files.  In particular, it assumes that the game
  or simulator is logging to a learning record store (essentially a
  database) event descriptions that look something like:
  
  \begin{algorithm}
    \caption{Generic Event Record, JSON format.}
    \label{json:event}
    \begin{listing}
      {
        app:"Assessment Name",
        uid:"Student/User ID",
        timestamp:"Time at which event occured",
        verb:"Action Keyword",
        object:"Object Keyword",
        data:{
          field1:"Value",
          field2:["list","of","values"],
          field3:{part1:"complex",part2:"object"}
        }
      }
  \end{listing}
  \end{algorithm}

  The format of Example~\ref{json:event} is JSON (java scrip object
  notation).  This is list of key--value pairs with the key and value
  separated by a colon (:).  The values can be numeric values,
  strings, date-time objects, arrays (using the square brackets, []),
  or objects (using curly braces, \{\}).  This event format is a
  simplified version of the xAPI\footnote{The simplification is mainly
    replacing the \emph{verb} and \emph{object} values with strings
    where xAPI uses more complex objects.  In particular, xAPI uses a
    full URL to uniquely define the verb and object, as the same word
    could have different meanings in the context of the application.
    In the simplified version, the vocabulary is defined by the
    application, allowing the message itself to be simpler.} format
  used by Learning Locker \cite{xAPI}.  The first five header fields
  are common to every event while the format of the data field is
  completely open and can contain arbitrarily complex status data.

  The assumption is that the the work product the EIP receives from
  the PP is an ordered sequence of these event records.  (This could
  either be directly sent from the PP to the EIP or the EIP could
  retreive them on demand from a learning record store.)  The EIP
  processes these one at a time, updating the state of the system.  In
  the end, the EIP must send one (or more) message to the EAP stating
  providing values for the observables seen in a particular context.
  This is what the EAP uses to update the student model.
  
  \section{Tasks and Contexts}
  \label{sec:context}




  \section{State Machine}
  \label{sec:stateMachine}

  \subsection{Observables}
  \label{sub:obs}
  
  \subsection{Flags and Timers}
  \label{sub:flags}

  \subsection{Events}
  \label{sub:events}

  \subsection{Dot notation}
  \label{sub:dot}

  \section{Rules}
  \label{sec:Rules}

  \subsection{Types and Timing}
  \label{sub:types}

  \subsection{Conditions}
  \label{sub:cond}

  \subsection{Predicates}
  \label{sub:pred}

  \subsection{Rule Set Maintenance}
  \label{sub:ruleSet}

  \section{Examples}
  \label{sec:examples}
  
  \section{Software}
  \label{sec:software}

  \bibliographystyle{apacite}
  \bibliography{EIRefs}
  
\end{document}
